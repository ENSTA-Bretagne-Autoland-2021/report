\section{Introduction}

La multiplication des champs éoliens offshore nécessite le développement de solutions intelligentes permettant leur inspection régulière et leur maintenance. Dans ce cadre, Forssea-Robotics développe un système de ROV autonome (Argos\cite{forssea}) capable d’effectuer des missions d’autonomes d’inspection de structures sous-marines. Ce ROV est déployé depuis un USV et opère en autonomie complète en remontant des informations utiles aux opérateurs. Puisque les champs éoliens comportent aussi des structures émergées, le développement d’un système d’inspection aérien couplé à l’USV permettrait au système d’être capable d'inspecter de façon totalement autonome tous les éléments de ces champs. L’une des difficultés majeures de ce genre de système autonome réside dans les manœuvres délicates de décollage et d'atterrissage, puisque deux robots doivent opérer en collaboration pour docker sans erreur.

Dans ce projet, nous proposons une solution pour faire atterrir de manière autonome un UAV (Unmanned Aerial Vehicle) sur un UGV (Unmanned Ground Vehicle) : le docking autonome. Les applications de ce projet sont de plus en plus nombreuses : flottes de drones, adaptation à un docking d’un AUV vers un USV …
Nous proposons une solution générique utilisant peu de capteurs. Un marqueur ArUco (motif avec pixels noirs et blancs) est posé sur l’UGV, l’UAV est équipé d’une caméra. Un algorithme de vision 3D permet la localisation de l’UAV dans le repère de l’UGV. Ensuite nous proposons deux méthodes pour commander l’UAV et le poser sur l’UGV. Une première méthode propose d’utiliser un champ de vecteur pour guider le drone et une deuxième méthode propose d’utiliser une méthode de Reinforcement Learning.
