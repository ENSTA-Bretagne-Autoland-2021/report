\section{Asservissement de l'UAV}

Les équations dynamiques définissant l’évolution d’un quadrotor sont connues. C’est pourquoi il est facile de simuler le comportement de ce robot. 
Nous choisirons ici de commander le drone sans se préoccuper de sa dynamique, 
c'est-à-dire que l’on va le contrôler en s’inspirant de la manière dont un pilote pourrait le faire.


On va donc devoir implémenter un contrôleur, qui en connaissant une position voulue pour le drone et sa position actuelle, 
va trouver les commandes à lui appliquer afin de l'amener dans cet état souhaité. Cette méthode de type (PID) présente 
l’avantage de ne pas avoir besoin de manipuler les équations d’état d’un quadrotor qui peuvent être compliquées, 
bien que connues, afin d’établir les commandes à lui appliquer. On aurait pu à ce titre utiliser une méthode de bouclage linéarisant. 
On verra que les résultats produits par notre méthode sont tout de même satisfaisants.



Nous devons donner à notre PID un vecteur vitesse désiré. 
Ce vecteur vitesse est calculé en phase d’approche par la méthode des potentiels artificiels. 
Il s’agit d’une analogie avec l’électromagnétisme. On considère la position souhaitée comme une charge électrique fictive 
produisant un champ électrique dans son voisinage nommé 
