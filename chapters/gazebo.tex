\section{Simulateur Gazebo}

Un simulateur est un outil important dans le développement d’une application robotique. En effet, il permet de tester les implémentations logicielles sans même avoir accès à du matériel. Il permet aussi de tester le robot rapidement dans différentes situations qui ne seraient pas facilement trouvables dans l’environnement réel. Enfin il permet de faire un bon nombre de tests qui sont souvent destructifs lors d’une phase d’apprentissage d’un réseau de neurones dans le cadre de l’implémentation d’algorithmes de Machine Learning. En outre le développement d’un simulateur ne substitue en rien la réalisation de tests réels après avoir testé les algorithmes implémentés en simulation.

Gazebo\footnote{\url{http://gazebosim.org/}} est un simulateur physique réputé pour être simple d’utilisation et configurable à souhait. Il est puissant et permet de simuler des systèmes physiques de manière poussée, qu’ils soient dans des milieux sous-marins, aériens ou terrestres. Cela nous laisse donc la possibilité de réaliser un simulateur pour notre application. L’avantage de Gazebo réside aussi dans le fait qu’il est interfaçable avec le Middleware ROS. Il nous permet donc de simuler le comportement de robots qui ont une implémentation logicielle ROS pour les faire naviguer dans l’environnement simulé. On peut ainsi valider le comportement des robots et l’implémentation logicielle dans le simulateur avant de lancer ces mêmes algorithmes sur les robots réels. C’est en quoi il est particulièrement intéressant de réaliser un simulateur dans le domaine de la robotique.

Pour ce faire, nous avons besoin de créer un monde Gazebo permettant la simulations de deux éléments de notre application : un drone de type quadrotor pilotable et ayant un comportement physique semblable à un drone grand public et une plateforme mobile présentant un Aruco permettant d'accueillir l’UAV. 

Pour ce qui est du drone, nous allons nous appuyer sur les travaux menés à l’Université Technique de Darmstadt sur la réalisation d’un simulateur de drone nommé Hector\footnote{\url{https://github.com/tu-darmstadt-ros-pkg/hector_quadrotor}}~\cite{Meyer2012ComprehensiveSO}. Ils ont proposé une simulation d’un drone en ROS et Gazebo open-source afin de permettre à quiconque de pouvoir rapidement construire un simulateur mettant en jeu un quadrotor. Ce drone comporte une caméra orientée vers le sol permettant de capter une image qui est publiée sur le réseau ROS. C’est donc un capteur se trouvant dans la même configuration que sur notre drone réel et qui nous permet de simuler son comportement dans l’environnement simulé Gazebo.

La plateforme mobile munie d’un marqueur est quant à elle réalisée simplement à base d’un cube sur lequel on a disposé un marqueur ArUco~\cite{kurosu2018human,campilho2018image}. La position du cube est changeable à souhait via l’API proposée par Gazebo et nous sommes aussi en mesure de piloter cette position via le réseau ROS. Cela permet de simuler simplement le comportement de l’UGV pour notre application, même si on aurait pu remplacer le cube par un UGV simulé avec un marqueur Arco positionné à son bord. Il aurait par contre fallu réaliser un nœud ROS permettant de piloter l’UGV afin de tester le docking dans différentes configurations.

Nous avons ainsi à notre disposition un environnement de simulation complet de notre application qui peut nous permettre de tester nos algorithmes, nos lois de commandes, mais aussi nous permettre de tenter d’implémenter des méthodes de Machine Learning qui nécessitent de mettre notre système dans de nombreux cas de figures afin de lui permettre d’apprendre à piloter votre drone.
