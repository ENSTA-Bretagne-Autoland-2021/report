\section{Conclusion}

Ce projet fut l’occasion de travailler sur des UAV, 
nous n'avions jamais travaillé sur ce type de robot. 
Malheureusement pour plusieurs raisons, à notre goût, 
nous n’avons pas assez expérimenté dans le monde réel. 
En contre-partie et avec la contrainte du travail en distanciel, 
nous avons travaillé en simulation et découvert l’outil Gazebo ainsi que perfectionner 
nos méthodes de travail en projet avec l’outil Git. 
Par ailleurs, nous avons obtenu une méthode robuste de localisation sans GPS 
à l’aide de marqueurs. Aussi, nous avons proposé un état de l’art sur la commande des UAV présents à l’école. La commande d’un UAV tel que le DJI Phantom 4 pourrait être un sujet complet de projet industriel, plusieurs compétences sont requises telles que la maîtrise du framework android, ainsi que des connaissances en commande de robots et ROS. Le modèle Matrix 600 ne permet pas de pratiquer en liberté du fait de la réglementation française. Au cours de l’avancée du projet les objectifs initiaux fixés ont pu évoluer avec principalement l’ajout d’un contrôleur issu de méthodes de renforcement. Ce contrôleur donne des résultats encourageants. Le contrôleur permet d’asservir le drone en position. Nous pensons qu’avec quelques améliorations comme par exemple l’ajout de réseaux neurones à  convolution pour traiter l’image brute, nous pourrions obtenir des résultats difficiles à obtenir avec des méthodes classiques de contrôle de robots, e.g. le docking d’un UAV sur un USV en mouvement à cause des vagues de houle. 