\section{Conclusion}

Ce projet fut l’occasion de travailler sur des UAV, robots avec lesquels nous n'avions jamais travaillé auparavent. Nous avons ainsi pu expérimenter des méthodes de commande de robots basée sur des champs de vecteurs, qui sont assez classiques dans le domaine de la robotique mobile, mais aussi avec des méthodes de Reinforcement Learning qui semblent être au c\oe ur des développements robotiques actuels. Par ailleurs, nous avons pu tester des méthode robuste de localisation sans GPS à l’aide de la bibliothèque ArUco. 

Aussi, nous avons proposé un état de l’art sur la commande des UAV présents à l’école. La commande d’un UAV tel que le DJI Phantom 4 pourrait être un sujet complet de projet industriel, plusieurs compétences sont requises telles que la maîtrise du SDK\footnote{Software Development Kit} android, ainsi que des connaissances en commande de robots et ROS. Le modèle Matrix 600 ne permet pas de pratiquer en liberté du fait de la réglementation française et de la sécurité. 

Au cours de l’avancée du projet les objectifs initiaux fixés ont pu évoluer avec principalement l’ajout d’un contrôleur issu de méthodes de renforcement. Ce contrôleur donne des résultats encourageants, et il semble intéressant de continuer des développements dans cette voie afin de résoudre des problèmes de dockings plus généraux. Nous pensons qu’avec quelques améliorations comme l’utilisation de réseaux neurones à  convolution pour traiter l’image brute de la caméra de l'UAV, nous pourrions obtenir des résultats difficiles à obtenir avec des méthodes classiques de contrôle de robots, avec comme application par exemple le docking d’un UAV sur un USV en mouvement à cause des vagues de houle.